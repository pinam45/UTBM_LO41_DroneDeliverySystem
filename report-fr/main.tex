\documentclass[article, backcover, french, nodocumentinfo]{upmethodology-document}
\include{packages}
% For more information about UPmethodology: https://www.ctan.org/pkg/upmethodology

%=======================================================================================================
%=============================================== Informations ==========================================
%=======================================================================================================

%%% Document Information and Declaration
\declaredocument{Simulation d'un système de livraison par drone}{Projet de LO41}{---}

%%% Abstract and Key-words
\setdocabstract[french]{Projet UTBM de l'UV LO41 du semestre de printemps 2017}
\setdockeywords[french]{UTBM, LO41, drone, delivery, system}

%%% Document Authors and Validators
\addauthorvalidator*[jerome.boulmier@utbm.fr]{Jérôme}{Boulmier}{Étudiant en INFO02}
\addauthorvalidator*[maxime.pinard@utbm.fr]{Maxime}{Pinard}{Étudiant en INFO02}

%%% Informed People
\addinformed*[philippe.descamps@utbm.fr]{Philippe}{Descamps}{Professeur de l'UV LO41}

%%% Copyright and Publication Information
\setcopyrighter{Jérôme Boulmier et Maxime Pinard}
\setpublisher{Jérôme Boulmier et Maxime Pinard}
\setprintingaddress{France}

%%% Version
\incversion{\makedate{\the\day}{\the\month}{\the\year}}{Initial version.}{\upmpublic}

%=======================================================================================================
%================================================== Configs ============================================
%=======================================================================================================

% Change Front Page Layout
%\setfrontcover{modern} % modern or classic

% Change Illustration Picture
%\setfrontillustration[1.3]{figures/figure}

% Source code formatting
\upmcodelang{cpp} % uml, java or cpp

% Prevent page breaks in paragraphs
\predisplaypenalty=1000
\postdisplaypenalty=1000
\clubpenalty=1000

% Minimal space required in the bottom margin not to move the title on the next page
%\renewcommand{\bottomtitlespace}{.1\textheight}

% Links config, especialy for the table of contents
\hypersetup{
    colorlinks=true,
    linkcolor=black,
    urlcolor=blue,
    linktoc=all
}

% French language config
\frenchbsetup{StandardLayout=true,ReduceListSpacing=false,CompactItemize=false}

%=======================================================================================================
%================================================= Functions ===========================================
%=======================================================================================================

%Paragraph with line break
\newcommand{\p}[1]{\paragraph{#1\\}}

% Function to print a warning sign
\newcommand{\dangersign}[1][2.5ex]
	{\renewcommand{\stacktype}{L}
		{\scaleto{\stackon[1pt]{\color{red}$\triangle$}{\fontsize{4pt}{4pt}\selectfont !}}{#1}}}

% Definition of some dt/dx/dy shortcuts for integrals
\newcommand{\dt}
{\;\mathrm{d}\,t}

\newcommand{\dx}
{\;\mathrm{d}\,x}

\newcommand{\dy}
{\;\mathrm{d}\,y}

% Definition of \Witem for 'itemize' environment with a warning sign
\newcommand{\Witem}
{\item[\dangersign{}]}

% Definition of a Max function shortcut
\newcommand{\Max}[2][ ]
{\underset{#1}{\text{Max}}\,#2}

% Definition of a Min function shortcut
\newcommand{\Min}[2][ ]
{\underset{#1}{\text{Min}}\,#2}


\newcommand{\TODO}[2][ ]{\todo[inline,color=green]{#2}}

\begin{document}
	\upmdocumentsummary{}
	\upmdocumentauthors{}
	%\upmdocumentvalidators{}
	\upmdocumentinformedpeople{}
	\upmpublicationpage{}
	\thispagestyle{empty}
	\tableofcontents{}
	%\lstlistoflistings{}
	%\listoffigures{}
	\newpage{}
	\section{Conception}
		\subsection{Réseau de Petri}
			\begin{figure}[H]
			  \centering
			  \includegraphics[width=\textwidth]{figures/petri_drones}
			  \caption{Réseau de Petri}
			  \label{fig:petrinet}
			\end{figure}
			\TODO{Explications}
		\subsection{Communication}
			\subsubsection{Threads et Mutex}
				\TODO{Threads mieux que Processus}
				\TODO{Mutex pour éviter accès concurrents, pas utilisés pour la synchronisation}
			\subsubsection{Files de messages}
				\TODO{Synchronisation + aspect messages de communication}
	\section{Réalisation}
		\subsection{Listes}
			\TODO{LinkedList, possibilité d'insertion triée pour colis par ordre de priorité\ldots}
		\subsection{Structures}
			\TODO{Fonction pour créé / alouer et supprimer / libérer, fonction pour lancer dans un thread\ldots}
			\subsubsection{Mothership}
			\subsubsection{Drone}
			\subsubsection{Client}
			\subsubsection{Package}
		\subsection{Interface utilisateur}
			\subsubsection{ConsoleControl}
				\TODO{lib pour faire des UI en console}
			\subsubsection{Tableau de bord (dashboard)}
				\TODO{Thread avec file de messages}
				\TODO{Affiche état de chaque acteur/élément du programme}
				\TODO{Configurable + expliquer config actuelle}
	\section{Utilisation}
		\subsection{Compilation}
			\subsubsection{Configuration}
				\TODO{pédantic}
				\TODO{Norme POSIX magique}
				\TODO{Prérequis: version min de gcc}
			\subsubsection{Makefile}
		\subsection{Exécution}
			\subsubsection{Paramètres}
				\TODO{Fichiers de config a passer en paramètres}
				\TODO{Fichiers par défaut et de test fournis}
			\subsubsection{Tests}
				\TODO{Valgrind}
\end{document}
