\documentclass[article, backcover, french, nodocumentinfo]{upmethodology-document}
\include{packages}
% For more information about UPmethodology: https://www.ctan.org/pkg/upmethodology

%=======================================================================================================
%=============================================== Informations ==========================================
%=======================================================================================================

%%% Document Information and Declaration
\declaredocument{Simulation d'un système de livraison par drone}{Projet de LO41}{---}

%%% Abstract and Key-words
\setdocabstract[french]{Projet UTBM de l'UV LO41 du semestre de printemps 2017}
\setdockeywords[french]{UTBM, LO41, drone, delivery, system}

%%% Document Authors and Validators
\addauthorvalidator*[jerome.boulmier@utbm.fr]{Jérôme}{Boulmier}{Étudiant en INFO02}
\addauthorvalidator*[maxime.pinard@utbm.fr]{Maxime}{Pinard}{Étudiant en INFO02}

%%% Informed People
\addinformed*[philippe.descamps@utbm.fr]{Philippe}{Descamps}{Professeur de l'UV LO41}

%%% Copyright and Publication Information
\setcopyrighter{Jérôme Boulmier et Maxime Pinard}
\setpublisher{Jérôme Boulmier et Maxime Pinard}
\setprintingaddress{France}

%%% Version
\incversion{\makedate{\the\day}{\the\month}{\the\year}}{Initial version.}{\upmpublic}

%=======================================================================================================
%================================================== Configs ============================================
%=======================================================================================================

% Change Front Page Layout
%\setfrontcover{modern} % modern or classic

% Change Illustration Picture
%\setfrontillustration[1.3]{figures/figure}

% Source code formatting
\upmcodelang{cpp} % uml, java or cpp

% Prevent page breaks in paragraphs
\predisplaypenalty=1000
\postdisplaypenalty=1000
\clubpenalty=1000

% Minimal space required in the bottom margin not to move the title on the next page
%\renewcommand{\bottomtitlespace}{.1\textheight}

% Links config, especialy for the table of contents
\hypersetup{
    colorlinks=true,
    linkcolor=black,
    urlcolor=blue,
    linktoc=all
}

% French language config
\frenchbsetup{StandardLayout=true,ReduceListSpacing=false,CompactItemize=false}

%=======================================================================================================
%================================================= Functions ===========================================
%=======================================================================================================

%Paragraph with line break
\newcommand{\p}[1]{\paragraph{#1\\}}

% Function to print a warning sign
\newcommand{\dangersign}[1][2.5ex]
	{\renewcommand{\stacktype}{L}
		{\scaleto{\stackon[1pt]{\color{red}$\triangle$}{\fontsize{4pt}{4pt}\selectfont !}}{#1}}}

% Definition of some dt/dx/dy shortcuts for integrals
\newcommand{\dt}
{\;\mathrm{d}\,t}

\newcommand{\dx}
{\;\mathrm{d}\,x}

\newcommand{\dy}
{\;\mathrm{d}\,y}

% Definition of \Witem for 'itemize' environment with a warning sign
\newcommand{\Witem}
{\item[\dangersign{}]}

% Definition of a Max function shortcut
\newcommand{\Max}[2][ ]
{\underset{#1}{\text{Max}}\,#2}

% Definition of a Min function shortcut
\newcommand{\Min}[2][ ]
{\underset{#1}{\text{Min}}\,#2}


\begin{document}
	\upmdocumentsummary{}
	\upmdocumentauthors{}
	%\upmdocumentvalidators{}
	\upmdocumentinformedpeople{}
	\upmpublicationpage{}
	\thispagestyle{empty}
	\tableofcontents{}
	%\lstlistoflistings{}
	%\listoffigures{}
	\newpage{}
	\section*{Introduction}\addcontentsline{toc}{section}{Introduction}
		\paragraph*{}
			Dans le cadre de l'UV LO41, nous devions réaliser un simulateur de livraison
			par drone inspiré de celui que Amazon essaye de mettre en place. Afin de répondre à ce problème, nous avons commencé par analyser le
			problème grâce à un réseau de Petri, par la suite nous avons réfléchis sur l'implémentation
			du problème en nous posant des questions sur la communications entre les différents
			acteurs du système de livraison. Après cela, nous avons réalisé l'implémentation et
			 avons fini par une analyse des performances de notre programme.
	\section{Conception}
		\subsection{Réseau de Petri}
			\begin{figure}[H]
			  \centering
			  \includegraphics[width=\textwidth]{figures/petri_drones}
			  \caption{Réseau de Petri}
			  \label{fig:petrinet}
			\end{figure}
			\paragraph*{}
				Le réseau de Petri de la figure \ref{fig:petrinet} décrit le problème du point de vue du drone, pour les besoins de la
				simulation du réseau de Petri, on a placé 60 drones étant donné qu'il y a 3
				cas où le drone peut s'écraser. Mais dans la réalité, il n'y a qu'une très
				faible chance que le drone ait un problème technique.
			\paragraph*{}
				Pour commencer à livrer un client, un drone et un paquet doivent être
				disponible, une fois le drone chargé, le drone s'envole pour livrer le
				client, on a alors trois choix disponibles:
				\begin{itemize}
					\item Le drone s'écrase
					\item Le client accepte le paquet
					\item Le client refuse le paquet
				\end{itemize}
				Dans les cas où le client accepte ou refuse le colis, le drone retourne vers le vaisseau mère,
				il peut également s'écraser	durant ce vol retour.
			\paragraph*{}
				Dans le cas ou le drone s'écrase avec un colis, le client
				devient un client ``pas livré'' car son colis n'existe plus, si le drone
				retourne au vaisseau mère avec le colis, le colis est remis dans la liste
				des livraisons et le drone est de nouveau disponible.
			\paragraph*{}
				Enfin lorsque le drone est disponible, s'il n'a plus de batterie, il peut
				aller se charger.
		\subsection{Communication}
			\subsubsection{Threads}
				\paragraph*{}
					Dans cette implémentions il a été choisi d'utiliser des threads plutôt que des processus.
					Cela permet entre autre d'éviter la surcharge lié à l'utilisation des processus.
					De plus les threads partagent leur espace d'adressage, ce qui permet une communication
					plus efficace entre les différents acteurs du programme (clients, drones, vaisseau mère).
			\subsubsection{Files de messages}
				\paragraph*{}
					Les files de messages sont les éléments centraux de ce programme, en plus de permettre aux drones,
					clients et au vaisseau mère de communiquer entre eux. Elles assurent également la synchronisation entre
					les différents acteurs. Par exemple lorsqu'un drone à fini de se charger il va envoyer un message au vaisseau
					mère afin de le prévenir puis se mettre en attente de la réception d'un message lui indiquant quoi faire. Le vaisseau mère a la réception
					du message va alors considérer le drone comme disponible et lui confier un paquet si les conditions sont réunies
					(paquet disponible, le drone peut porter le paquet\ldots) et lui envoyer un message pour lui dire de partir livrer le client.
			\subsubsection{Mutex}
				\paragraph*{}
					Les mutex sont utilisés pour éviter des accès concurrents, ils permettent de protéger des ressources critiques,
					en effet, on ne protège que les variables qui sont changés par un thread et qui peuvent être lu par d'autres threads.
	\section{Réalisation}
		\subsection{Structures}
			\paragraph*{}
				Dans ce projet, nous avons réalisé des fonctions qui prennent en charge la création et la suppression des différentes structures de données.
				Ceci afin de rendre le code plus lisible et d'éviter toute fuite de mémoire du a une structure mal libérée.
				De plus les drones, les clients, le mothership et le dashboard ont une fonction permettant de les lancer dans un thread.
			\subsubsection{Mothership}
				\paragraph*{}
					Au début du programme le mothership détient la liste des paquets, des clients et des drones.
					Le mothership est le thread principal du programme, c'est lui qui va lancer la simulation et par la suite va s'occuper de
					la communication avec les drones, à la fin de la simulation, il va aussi notifier les clients que la livraison des paquets
					restant est impossible (plus de drones disponible capable de porter les paquets restant ou l'on a essayé de livrer le paquet 3 fois).
					Puis il prévenir tout les drones encore vivant de la fin de la journée de livraison avant de rentrer a la base, c'est a dire mettre
					fin a la simulation de livraison.
			\subsubsection{Drone}
				\paragraph*{}
					Le drones ont tous une autonomie, une charge maximum et un temps de recharge. Ils ont également un pointeur sur le client a livrer,
					et un pointeur sur le colis a livrer.
				\paragraph*{}
					Les drones vont communiquer avec le client et le vaisseau mère, le vaisseau mère prévient le drone de ce qu'il doit faire par message (aller livrer, se recharger ou s'éteindre). Lorsqu'il il va livrer le drone va prévenir le client de son arrivée pour que celui-ci sorte la cible. Si le client sort la cible, la livraison réussie, le drone prévient le vaisseau mère par message puis rentre au vaisseau sans le colis. Si le client est absent et ne sort pas sa cible, la livraison échoue, le drone prévient le vaisseau mère par message puis retourne au vaisseau mère avec le colis. Une fois arrivé au vaisseau mère le drone prévient le vaisseau mère de son arrivé par message, ce dernier lui assigne alors une autre livraison ou l'envoi se recharger, dans le cas du retour d'une livraison qui a échouée, le nombre d'essais restant du colis est décrémenté et le colis est remis dans la liste des livraisons.
				\paragraph*{}
					Les drones peuvent avoir une panne technique qui les fait s'écraser durant le voyage allé ou retour, si ces dernier transportent un colis au moment du crash le colis est perdus.
			\subsubsection{Client}
				\paragraph*{}
					Le client a comme attribut principal la distance qui le sépare au vaisseau mère. Le client écoute uniquement des messages,
					il se charge de sortir ou non (pour simuler des absences) sa cible pour que le drone puisse trouver le point d’atterrissage.
					Il reçoit également son paquet du drone.
				\paragraph*{}
					Un client a finit sa journée une fois qu'il a reçut tout les paquets qu'il attendait ou lorsque le vaisseau mère le préviens qu'il ne recevra pas les colis restants (pas de drones capable de le transporter, colis détruit par la mort d'un drone, nombre d'essais atteint).
			\subsubsection{Package}
				\paragraph*{}
					La structure package représente le colis à livrer au client, elle possède un poids, une priorité et l'identifiant du client à qui livrer le colis.
					Les colis prioritaires sont livrés les premiers. Les colis peuvent être perdu dans le cas où le drone s'écrase en les transportant.
			\subsubsection{Listes}
				\paragraph*{}
					La liste est une liste doublement chaînée, permettant de stocker les informations des différents éléments du programme.
					Elle nous permet d'insérer la où l'on veut et notamment d'utiliser une insertion triée, ce qui permet de garder les paquets
					triés par priorité dans la liste, de ce fait le paquet le plus prioritaire sera livré en premier sauf s'il n'y a pas de drone
					capable de le livrer, dans le cas échéant, on essayera de livrer le paquet suivant s'il existe.
		\subsection{Interface utilisateur}
			\subsubsection{ConsoleControl}
				L'interface utilisateur est entièrement en console mais emprunte beaucoup aux interfaces graphiques, elle a été réalisée avec la librairie C, \textbf{ConsoleControl} développée par Maxime Pinard, membre du groupe du projet. Le projet utilise la version 0.2 de la librairie.
				\paragraph*{}
					Fonctionnalités principales:
					\begin{itemize}
						\item Obtention d'informations sur la console (largeur, hauteur\ldots)
						\item Positionnement du curseur
						\item Changement des couleurs d'arrière-plan et de premier plan
						\item Gestion des inputs
						\item Dessin géométrique (lignes, rectangle)
						\item Interface utilisateur:
							\begin{itemize}
								\item Menu
								\item Menu d'options
								\item Messages
							\end{itemize}
						\item \ldots
					\end{itemize}
				\paragraph*{}
					Avantages:
					\begin{itemize}
						\item Pas de dépendances
						\item Utilisé en sous module compilé en même temps que le projet
					\end{itemize}
				\paragraph*{}
					Pour plus d'informations, voir \href{https://github.com/pinam45/ConsoleControl}{la page Github de la librairie}.
			\subsubsection{Tableau de bord (dashboard)}
				\p{Lancement}
					Le dashboard est représenté par la structure \jclass{Dashboard}, pour faire fonctionner ce dernier il faut allouer et la structure a l'aide de la fonction \jfunc{dashboard\_constructor} et lancer la fonction \jfunc{dashboard\_launch} avec en paramètre la structure. Dans le projet le dashboard est lancer dans un thread qui lui est réservé.
				\p{Communication}
					Pour mettre a jour les information du dashboard un système de communication par file de messages est disponible, pour cela il suffit d'utiliser une structure \jclass{DashboardMessage} et la fonction \jfunc{dashboard\_sendMessage}. La structure \jclass{DashboardMessage} contient le type d'élément concerné (drone, paquet, client), sont identifiant (id) et son nouvel état.
				\p{États}
					\begin{figure}[H]
						\centering
						\includegraphics[width=\textwidth]{figures/UI1}
						\caption{Interface utilisateur: en cour d’exécution}
						\label{fig:UIrunning}
					\end{figure}
					Plusieurs états sont possible pour chaque type d'élément, exemples avec la figure \ref{fig:UIrunning}:
					\begin{itemize}
						\item Le paquet 000 a bien été livré
						\item Le paquet 001 est en vol
						\item Le paquet 002 est en attente d’être livré
						\item Le drone 000 est en vol vers le client
						\item Le drone 001 est en charge
						\item Le drone 005 est en attente (soit il attend qu'un paquet lui soit attribué soit il attend la fin de journée car les paquets restant sont trop lourd ou les clients trop loin pour lui)
						\item Le drone 007 est mort suite a un incident technique
						\item Le drone 009 est en vol vers le vaisseau mère après une livraison réussie
						\item Le client 000 est absent lors de sa livraison
						\item Le client 001 attend sa livraison
						\item Le client 002 a reçut tout ces paquets
						\item Le client 019 a sortit sa cible pour l'arrivée du drone
					\end{itemize}
					Tout au long de l’exécution du programme les différents acteurs du programme envoient des messages au dashboard pour prévenir de leur état jusqu’à la fin du programme ou chaque acteur est dans un état terminal comme visible sur la figure \ref{fig:UIend}.
					\begin{figure}[H]
						\centering
						\includegraphics[width=\textwidth]{figures/UI2}
						\caption{Interface utilisateur: fin d’exécution}
						\label{fig:UIend}
					\end{figure}
					Les états finaux sont:
					\begin{itemize}
						\item Pour les paquets: livraison réussit ou échouée
						\item Pour les drones: journée finit ou mort
						\item Pour les clients: finit avec ayant reçut tout ces paquets ou finit avec des paquets manquants (limite d'essai atteint ou détruit avec la mort d'un drone)
					\end{itemize}
				\p{Configuration}
					Le dashboard est configurable, tout les messages d'état et les couleurs sont changeables. De plus le dashboard s'adapte dynamiquement a la taille de la console qui peut être redimensionnée au cour de l’exécution. En fonction de la largeur de la consule les élément s’afficheront sur 3 ou 1 colonne.
	\section{Utilisation}
		\subsection{Compilation}
			\subsubsection{Configuration}
				\paragraph*{Flags utilisé (gcc)}
					\begin{itemize}
						\item Flags de version du standard:
							\begin{lstlisting}[breaklines=true,breakatwhitespace=true,breakindent=0pt,columns=fixed,keepspaces=true,frame=single,basicstyle=\footnotesize\sffamily]
-std=c11\end{lstlisting}
							Le projet utilise le standard \textit{ISO/IEC 9899:2011}, aussi connu sous le nom \textit{C11}.
						\item Flags de linkage des librairies:
							\begin{lstlisting}[breaklines=true,breakatwhitespace=true,breakindent=0pt,columns=fixed,keepspaces=true,frame=single,basicstyle=\footnotesize\sffamily]
-lConsoleControl -pthread -lrt\end{lstlisting}
							Le projet utilise la librairie \textit{ConsoleControl}, les thread POSIX de la librairie pthread et les files de message POSIX de la librairie \textit{rt} (POSIX.1b Realtime Extension).
						\item Flags de version POSIX:
							\begin{lstlisting}[breaklines=true,breakatwhitespace=true,breakindent=0pt,columns=fixed,keepspaces=true,frame=single,basicstyle=\footnotesize\sffamily]
-D_POSIX_C_SOURCE=200112L\end{lstlisting}
							On définit \textit{\_POSIX\_C\_SOURCE} pour utiliser la version \textit{200112L} de la norme POSIX.
						\item Flags de warning:
							\begin{lstlisting}[breaklines=true,breakatwhitespace=true,breakindent=0pt,columns=fixed,keepspaces=true,frame=single,basicstyle=\footnotesize\sffamily]
-pedantic -pedantic-errors -Wall -Wcast-align -Wcast-qual -Wconversion -Wdisabled-optimization -Wdouble-promotion -Wextra -Wfloat-equal -Wformat -Winit-self -Winvalid-pch -Wlogical-op -Wmain -Wmissing-declarations -Wmissing-include-dirs -Wpointer-arith -Wredundant-decls -Wshadow -Wswitch-default -Wswitch-enum -Wundef -Wuninitialized -Wunreachable-code -Wwrite-strings\end{lstlisting}
							Pour plus d'information voir les \href{https://gcc.gnu.org/onlinedocs/gcc/Warning-Options.html}{options de warning gcc}.
					\end{itemize}
			\subsubsection{Makefile}
				Le \textit{Makefile} se charge de compiler d'abord la librairie \textit{ConsoleControl} puis de compiler le projet en liant cette dernière. Un \textit{Makefile} s'utilise avec le programme Make, pour plus d'informations sur Make voir \href{https://www.gnu.org/software/make/}{le site de gnu}.\\
				Le \textit{Makefile} se configure grâce aux variables définies dans les deux premières sections, la configuration de base fournie est celle pour un OS Linux standard utilisant les commandes du shell et gcc.
				\begin{upmcaution}
					Compiler en \textbf{debug} pour tester est déconseillé car le logeur et le dashboard vont s'afficher en même temps, rendant le tout illisible.
				\end{upmcaution}
				\p{Compilation}
					Plusieurs cibles sont définies:
					\begin{description}
						\item[help] Affiches les cibles définit et leur effet
						\item[silent] Cible par défaut, équivalent a \texttt{make --silent all}
						\item[all] Compile le projet
						\item[debug] Compile le projet en debug en activant le logeur
						\item[clean] Supprime tous les fichiers et dossiers générés par le \textit{Makefile}
					\end{description}
					Pour appeler une cible:
					\begin{lstlisting}[breaklines=true,breakatwhitespace=true,breakindent=0pt,columns=fixed,keepspaces=true,frame=single,basicstyle=\footnotesize\sffamily]
$ make <cible>\end{lstlisting}
					Pour simplement compiler le projet:
					\begin{lstlisting}[breaklines=true,breakatwhitespace=true,breakindent=0pt,columns=fixed,keepspaces=true,frame=single,basicstyle=\footnotesize\sffamily]
$ make\end{lstlisting}
					Pour supprimer tous les fichiers et dossiers générés par le \textit{Makefile}:
					\begin{lstlisting}[breaklines=true,breakatwhitespace=true,breakindent=0pt,columns=fixed,keepspaces=true,frame=single,basicstyle=\footnotesize\sffamily]
$ make clean\end{lstlisting}
		\subsection{Exécution}
			\subsubsection{Paramètres}
				\paragraph*{}
					Les drones, paquets et clients utilisé par le programme sont changeables, ils sont décrit dans des fichiers au format \textit{csv}. Si aucun n'arguments n'est passé au programme a son exécution les fichier par défaut (\textit{packages1.csv}, \textit{drones1.csv}, \textit{clients1.csv}) seront utilisés.
				\paragraph*{}
					Pour lancer le programme avec les fichiers par défault:
					\begin{lstlisting}[breaklines=true,breakatwhitespace=true,breakindent=0pt,columns=fixed,keepspaces=true,frame=single,basicstyle=\footnotesize\sffamily]
$ ./build/bin/UTBM_LO41_DroneDeliverySystem.elf\end{lstlisting}
					Pour lancer le programme avec des autres fichiers:
					\begin{lstlisting}[breaklines=true,breakatwhitespace=true,breakindent=0pt,columns=fixed,keepspaces=true,frame=single,basicstyle=\footnotesize\sffamily]
$ ./build/bin/UTBM_LO41_DroneDeliverySystem.elf PACKAGES_FILE CLIENT_FILE DRONE_FILE\end{lstlisting}
				\paragraph*{}
					3 groupes de fichiers de configurations de test sont fournis:
					\begin{enumerate}
						\item \textit{packages1.csv}, \textit{drones1.csv}, \textit{clients1.csv}
						\item \textit{packages2.csv}, \textit{drones2.csv}, \textit{clients2.csv}
						\item \textit{packages3.csv}, \textit{drones3.csv}, \textit{clients3.csv}
					\end{enumerate}
			\subsubsection{Tests}
				\p{Memory leaks}
					On utilise l’outil de profilage Valgrind pour tester la présence de memory leaks.\\
					On exécute la commande:
					\begin{lstlisting}[breaklines=true,breakatwhitespace=true,breakindent=0pt,columns=fixed,keepspaces=true,frame=single,basicstyle=\footnotesize\sffamily]
$ valgrind --leak-check=yes ./build/bin/UTBM_LO41_DroneDeliverySystem.elf\end{lstlisting}
					Et on obtient:
					\begin{lstlisting}[breaklines=true,breakatwhitespace=true,breakindent=0pt,columns=fixed,keepspaces=true,frame=single,basicstyle=\footnotesize\sffamily]
==18446==
==18446== HEAP SUMMARY:
==18446==     in use at exit: 0 bytes in 0 blocks
==18446==   total heap usage: 9,553 allocs, 9,553 frees, 190,043 bytes allocated
==18446==
==18446== All heap blocks were freed -- no leaks are possible
==18446==
==18446== For counts of detected and suppressed errors, rerun with: -v
==18446== Use --track-origins=yes to see where uninitialised values come from
==18446== ERROR SUMMARY: 0 errors from 0 contexts (suppressed: 0 from 0)\end{lstlisting}
					Nous n'avons donc pas de memory leak durant l'exécution du programme.
				\p{Utilisation de la mémoire RAM}
					De plus en exécutant la commande:
					\begin{lstlisting}[breaklines=true,breakatwhitespace=true,breakindent=0pt,columns=fixed,keepspaces=true,frame=single,basicstyle=\footnotesize\sffamily]
$ valgrind --tool-massif ./build/bin/UTBM_LO41_DroneDeliverySystem.elf \end{lstlisting}
					On obtient le graphique de le figure \ref{fig:MemoryUsage}.
					\begin{figure}[H]
						\centering
						\includegraphics[width=\textwidth]{figures/memory}
						\caption{Utilisation de la mémoire RAM au cours de l'exécution du programme}
						\label{fig:MemoryUsage}
					\end{figure}
					\paragraph*{}
						Le premier pic local correspond à l'ouverture des fichiers de configurations,
						et le deuxième pic local qui est le pic global correspond au lancement des threads (20,3KiB)
						On remarquera que plus de la moitié de la ram est utilisé par le système pour ouvrir les fichiers et pour lancer les threads.
						Cependant en utilisant des processus, une bonne partie de la ram utilisé aurait été dupliquée, ce qui aurait probablement augmenté la quantité total de
						ram utilisée.
	\section*{Conclusion}\addcontentsline{toc}{section}{Conclusion}
		\paragraph*{}
			Ce projet nous a permis d'approfondir nos connaissances sur la programmation système sous Linux, et de manière générale
			sous Unix grâce à l'utilisation de la norme POSIX qui nous permet d'avoir un programme plus multiplateforme et plus moderne
			dans la conception de l'API.
		\paragraph*{}
			Enfin, nous avons également eu l'occasion d'analyser notre projet après sa réalisation pour juger de l'efficacité des outils fournis par le système.
\end{document}
